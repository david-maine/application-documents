\subsection*{Essay}\label{sec:essay}

\subsubsection*{Instruction}
\textit{explaining your motivation for choosing the programme and indicating your specific interests within Mathematics.}

\subsubsection*{Resource Links}
\begin{itemize}
    \item https://www.mastersportal.com/articles/2093/motivation-letter-example-applying-to-a-masters-in-international-information-systems.html
    \item 
\end{itemize}

Dear Amissions Office,\\
\subsection*{Intro}
I am applying for a position in the Mathematics Masters of Science program at ETH Zurich. I wish to deepen my knowledge in the fields of Statistics and Computer Science with the aspiration to make novel contributions.\\
\\
\subsection*{Reasons for applying}
My ambition to study at ETH Zurich stems from my desire to challenge and develop my fundemental knowledge and research skills. ETH Zurich has an impressive history and a reputation for being academically rigorous, particulary in the scientific fields. I believe I would thrive in this type of academic environment.\\
\\
The one year component of in depth studies in the masters programs appeals to me as I wish to develop further specialised knowledge in the field of Mathematics and Statistics. The requirement profile is another aspect influencing my application, as I believe my missing prerequisites will cover some gaps current knowledge. I would then enjoy the \\
\\
I chose to apply for the Mathematics over the Computer Science program as I wish to research conceptual process, structures and techniques that can better identify these concepts within language. I would like to focuss on the underlying mathematics as I believe a robust understanding is the most effective way to make advances in the field. ETH Zurichs' reputation for focusing on the theoretical also appeals to me ...
\\
Switzerland is a diverse nation and I admire the inclusiveness and knowledge sharing such a society encourages.  I would be honoured to be surrounded by the calliber of student and faculty at ETH Zurch.\\
\\
My passion for mathematics stems from the satisfaction of relating a seemingly complex and unintuitive proplem or concept to a simple set of clear and often intuitive arguments.\\
\subsection*{Preparation}
I chose a dual undergraduate degree in Statistics and Computer Science for their similarity in the elagance and power of efficent and intuitive design. Mathematics is the backbone of Computer science and believe I have the rare ability to see how this relationship works. The order of mathematics. Generalizibility of Mathematics and Computer Science.\\
\\
I have taken the time and opportunities throughout my undergraduate degree to determine my professional ambitions and desires. I have experimented with positions and travelled.\\
\\
During my undergraduate degree I have worked in an industry team responsible for credit risk Statistical modelling. This has  given me a strong sense of application. Though abstractly it has also encouraged a pattern of thinking often disregarded in practical application . In A highly regulated environment, a model that seems to work, or is good enough is not acceptable. A solution is only valid if it is explainable. I was also surrounded by team members with extensive in varying disciplines. \\
\\
My experience in the Credit Risk Analytics team gave me early understanding of the applications of statistics and mathematics towards practical use. Though I believe it is important to understand and appreciate the use cases of mathematics, I also learnt that I did not want to follow a career of pure application and knew my passion was for learning and researching.\\ 
\\
In my recent position as a research assistant at Oracle Labs I feel confident in pursuing a research career. I am working a project that aims to build a graph based knowledge network of security vulnerabilites in open source software. This encompasses many challenges such as analysing a large number of heterogenous sources (issue trackers, version control commits, official vulnerability databases, online forums, etc) to determine a number of homogenous properties (vulnerability, softare, versioning, etc.). The most challenging research component of this project is identifying the changes in software releases that represent patches or feature releases. We can try to differentiate this by analysing changes in source code, or a number of expressive text such as issue trackers (GitHub, JIRA, etc), git commits and online forums. In this project I am particularly interested developing my skills in the current state of the art Named Entity Recognition approaches.\\
\subsection*{Evidence of Skillset}
I have performed well in my academics, achieving a high GPA of 6.864 out of 7 to date, and a perfect 7 for the mathematics component of my degree.\\
\\
Subjects that have I have received a distiction (6) opposed to high distinction (7) have been less related to the logic and ... of mathematics and computer science that I enjoy. These have focused on topics such as social impacts or management strategies that though I acknowledge as an important skillset and do believe I developed as a reult, did not encourage me to passionately engage with the content as I have in my other coursework.
\\
I also believe I have a rare combination of study in both mathematics and computer science.\\
\\
The more practical side of my computer science degree also makes me a more productive mathematician and researching. I feel more comfortable writing quality and robust programs following professional standards, something I often observed as lacking when my mathematics peers are developing. Whether it be implementing an algorithm or running simulations this crossover of knowledge makes my work efficient.\\
\\
For example my current work as a research assistant Oracle Labs involves researching and performing analysis, though I can also communicate with a software engineer and follow best practices to reduce the work of transitioning from a development to production solution.\\
\subsection*{Goals}
I have a specific interest in advanced learning models, combining my Mathematics and Computer Science background. Specifically I am interested Computational Bayesian statistical methods, or tackling the problem of generalisation in learning models, particularly Natural Language Processing.
\\
Natural Language processing techniques. The concept of combining tradional machine learning with absract data structures of varying knowledge representation fascinates me. So much of the data that is available yet uninterpretable today age is text based. Text, representing human language, is also the medium that best conveys human thought. Fo this reason I see it as the most complicated, yet also most valuable data that can be analysed. I view the problem linked to a classical statistics problem; while there may always be stochastic noise or subjectivity to the problem, methods can be employed to better estimate the deterministic  I aim to build my knowledge and skills in this field to the point where I can engage with this scientific community making discoveries and make meaningful contributions myself.\\
\\
I would love to further research how abstruct structures and representations of unstructured data can be used to improve \\
\\
I intend to complete a Doctorate following from research beginning in my Masters Degree.\\
\subsection*{Conclusion}

In many ways language interpretation has parallel concepts to mathematics. Concepts of assignment, predicates, invariants, relationships can often be identified with a passage of text.\\
\\
My interests definitely span the fields of both mathematics and computer science. I am applying for the Mathematics program as I beleive the fundemental nature of thinking in appraoching theoretical mathematics problems is essential for making novel developments and advances.\\