Dear Admissions Office,\\
\\
\\
I am applying for a position in the Mathematics MSc program at University of Toronto. I am a dedicated student wishing to extend upon my knowledge in the fields of Statistics and Computer Science, with the aspiration to make novel and collaborative contributions to the research community.\\
\\
I hope to study at Univerity of Toronto for its impressive history and reputation as an academically rigorous institution, particularly in the scientific fields. I believe I would thrive in an environment surrounded by the caliber of students and faculty at University of Toronto. Toronto as a destination is attractive to me for postgraduate studies due to its diversity and inclusiveness. The combination of coursework and research project in the masters program appeals to me as I wish to develop further specialised knowledge in the field of Mathematics and Statistics. I would also hope to continue my studies into the Ph.D program to develop a major thesis.\\
\\
I am studying Statistics and Computer Science for my interest in the power of efficient and intuitive design. I have achieved high standards in my academics, achieving a GPA of 6.88 out of 7 to date, and a perfect 7 for the mathematics component. I have taken time and opportunities to determine my ambitions and motivations, working in industry, volunteer, and research positions. I have also undertaken fulfilling academic exchanges to Japan and France.\\
\\
I have worked as a Data Analyst for a Credit Risk statistical modelling team, where I developed a strong appreciation for the applications of statistics. In my current position as a Research Assistant at Oracle Labs, I am part of a project to build a graph based knowledge network of open source software vulnerabilities. This involves analysing a large array of data sources, from structured to natural language sources such as issue tracker fields and commit messages. Inferring information (vulnerabilities, software, versions) from these natural language sources rely on techniques such as Named Entity Recognition. Complementing theoretical analysis, I can use my skills in big data analytics to build current and customised datasets efficiently.\\
\\
My current research project's objective is to detect the presence of security patches in Java applications. My approach is to explore statistical and machine learning techniques to identify sub graphs representing software pathes within application call graphs generated from the compiled bytecode. These statistical techniques are advantageous over exact pattern matching to identify sub graphs similar to the search graph. These variations occur since the generated call graphs representative of software patches can differ between aplications due to the modification of surrounding code.\\
\\
\subsection*{Intent}
I have a specific interest in advanced learning models, which leverages my Mathematics and Computer Science background. In a broad sense I am interested to investigate the problem of generalisation in learning models, particularly for deep learning techniques.\\
\\
I am also particulary interested in graph based inference techniques, and how we can better utilise the inherent relationships graph based data structures represent. I have recently become interested in the concept of Graph Neural Network models to perform learning and inference on graph data structures. These are one of the techniques I hope to explore within my research project a Oracle Labs. Other considerations include spectral graph theory analytical tehniques, relying on matrix representations, though are less scalable.\\
\\
I also have an interest in Bayesian statistical theory and computational methods for intractable likelihoods and am interested to understand if the combination of variational inference with Graph Nueral Networks is beneficial.\\
\\
Longer term research objectives I would potentially like to investigate are open proplems in the field of Graph Neural Networks. One limitation is the convolutional approach that prevents a deep structure due to the stacking of multiple layers over smoothing the graph dataset. Another interesting problem is the extension of Graph Neural Netwrk models to dynamic graphs, where the model can adapt to changing nodes and relationships.\\
\\
I am applying for the Mathematics rather than the Computer Science program, as I wish to focus on the fundamental theory and concepts. I believe creative mathematics is the most promising approach to make significant advancements in these modelling techniques.\\
\\
I am a motivated and curious individual who would capitalise on every opportunity University of Toronto has to offer. I believe my background, abilities, and character makes me ideally suited for your Masters of Science program. Thank you for considering my application.\\
\\

Kind regards,\\
David Maine